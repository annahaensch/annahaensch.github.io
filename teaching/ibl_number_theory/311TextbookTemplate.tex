% Latex Template For Team Textbook

% !!!!!!!!!!!!   INSTRUCTIONS  !!!!!!!!!!!!!!

% 1) please use  Chapter_Number(Team).tex for your texfile, e.g. if you are on Team A have Chapter 1 use Chapter_1A.tex.

% 2) if you label a theorem or equation, use your chapter and team to identify the label such as 
% \label{aA_T1} for your first theorem if you are on Chapter 1 of Team A.

% 3) Do not use \def or \newcommand unless really necessary. If you can't help yourself,
% then do a find/replace throughout your section with the original non-macro version of 
% the command. 

% 4) There is a sample Chapter below. For individual chapter submissions just 
% cut and paste that template into you space below your chapter.  

% 5) Submit your work in both tex and pdf format, but make sure it compiles first. 



\documentclass[12pt]{article}
\usepackage{amssymb,amsmath,amsthm}
\usepackage{esint}
\usepackage{mathrsfs}
\usepackage{comment}
\usepackage{bbm,dsfont}


\newtheorem{theorem}{Theorem}
\newtheorem{lemma}[theorem]{Lemma}
\newtheorem{corollary}[theorem]{Corollary}
\newtheorem{conjecture}[theorem]{Conjecture}
\newtheorem{definition}[theorem]{Definition}
\newtheorem{example}{Example}
\newtheorem{remark}[theorem]{Remark}
\newtheorem{proposition}[theorem]{Proposition}


\usepackage{graphicx}

\def\N {{\mathbb N}}
\def\Z {{\mathbb Z}}
\def\Q {{\mathbb Q}}
\def\R {{\mathbb R}}
\def\C {{\mathbb C}}


\DeclareMathOperator{\GL}{GL}
\DeclareMathOperator{\SL}{SL}
\DeclareMathOperator{\SO}{SO}
\DeclareMathOperator{\Orthog}{O}
\DeclareMathOperator{\Sp}{Sp}
\DeclareMathOperator{\End}{End}
\DeclareMathOperator{\ord}{ord}
\DeclareMathOperator{\real}{Re}
\DeclareMathOperator{\imag}{Im}


%~~~~~~~~~~~~~~~~~~~~~~~~~~~~~~~~~~~~~~~~~~~~~~~~
%~~~~~~~~~~~~~~~~ Header and Table of Contents ~~~~~~~~
%~~~~~~~~~~~~~~~~~~~~~~~~~~~~~~~~~~~~~~~~~~~~~~~~

% For your individual chapter submissions you can ignore the header and table of contents. 
% This part will only become important when you compile all of the chapters.  
% Then you can uncomment this part and fill in the relevant details.  

% \title{Our Number Theory Textbook} % I encourage you to come up with a better title.  

% \author{P. Fermat\\  %Replace with your team member names, in alphabetical order.
% C. Gauss\\
% A. Haensch\\
% D. Hilbert\\
% J. Lagrange\\
% J. Liouville\\} 

% \date{December, 2015}

\begin{document}

% \maketitle

%\newpage
%\tableofcontents

%\newpage

% From here you should just be able to cut and paste your individual chapter contents into the appropriate space.  

%~~~~~~~~~~~~~~~~~~~~~~~~~~~~~~~~~~~~~~~~~~~~~~~~
%~~~~~~~~~~~~~~~~ Chapter 1 ~~~~~~~~~~~~~~~~~~~~~~~
%~~~~~~~~~~~~~~~~~~~~~~~~~~~~~~~~~~~~~~~~~~~~~~~~

\section{Divisiblity} % Replace this with your chapter number and title. 

\setcounter{equation}{0}
\setcounter{theorem}{0}

\addcontentsline{toc}{subsection}{{\it A. Haensch}} % Replace this with your name and chapter title. 
\begin{center}
{\it A. Haensch} % replace this with your name
\end{center}

\subsection{Introduction}

Here you will give a one-page introduction to your chapter highlighting key concepts, results, and techniques that will appear in the chapter.  This should be a way to tie the terms, theorems, and examples together. 

\subsection{Glossary of Terms}

In this section you will define your terms using the definitions we crafted throughout the course.

\begin{itemize}
\item An {\em axiom} is a statement or proposition that is regarded as being established, accepted, or self-evidently true.

\item A {\em theorem} is a general proposition not self-evident but proved by a chain of reasoning; a truth established by means of accepted truths.
\end{itemize}

\subsection{Theorems}

In this section you will state theorems from the course and give proofs.  Generally, these will looks something like the following. 

\begin{theorem}
The sum of two odd numbers is always even. 
\end{theorem}

\begin{proof}
Suppose that $a$ and $b$ are two arbitrary odd numbers, so by definition $x=2k+1$ and $y=2m+1$ for some $k,m\in \Z$.  Then,
\begin{eqnarray*}
a+b&=&(2k+1)+(2m+1)\\
&=& 2(k+m)+1+1\\
&=&2(k+m)+2\\
&=&2(k+m+1).
\end{eqnarray*}
But $k+m+1\in\Z$, and therefore $a+b$ is two times an integer and hence is even, by the definition of even. 
\end{proof}

\subsection{Examples}

Here you will give {\em at least} three nice examples, enumerated as below.  

\begin{example} This is your first example.  Here you might talk about $m\in \Z$ and  $n\in \N$ with
\[
m\equiv n\mod p,
\]
for some prime $p$. 
\end{example}

\begin{example}  This is your second example.  Here you might talk about a set of numbers, 
\[
\{0,1,2,...,p-1\}.
\]

\end{example}

\begin{example}  This is your third example.  Here you might do the Euclidean algorithm, to establish that 
\[
\gcd(a,b)=1.
\]

\end{example}

%~~~~~~~~~~~~~~~~~~~~~~~~~~~~~~~~~~~~~~~~~~~~~~~~
%~~~~~~~~~~~~~~~~ Chapter 2 ~~~~~~~~~~~~~~~~~~~~~~~
%~~~~~~~~~~~~~~~~~~~~~~~~~~~~~~~~~~~~~~~~~~~~~~~~

\section{Primes} % Replace this with your chapter number and title. 

\setcounter{equation}{0}
\setcounter{theorem}{0}

\addcontentsline{toc}{subsection}{{\it C. Gauss}} % Replace this with your name and chapter title. 
\begin{center}
{\it C.F. Gauss} % replace this with your name
\end{center}

\subsection{Introduction}

\subsection{Glossary of Terms}

\subsection{Theorems}

\subsection{Examples}

%~~~~~~~~~~~~~~~~~~~~~~~~~~~~~~~~~~~~~~~~~~~~~~~~
%~~~~~~~~~~~~~~~~ Chapter 3 ~~~~~~~~~~~~~~~~~~~~~~~
%~~~~~~~~~~~~~~~~~~~~~~~~~~~~~~~~~~~~~~~~~~~~~~~~

\section{Modularity} % Replace this with your chapter number and title. 

\setcounter{equation}{0}
\setcounter{theorem}{0}

\addcontentsline{toc}{subsection}{{\it J. Liouville}} % Replace this with your name and chapter title. 

\begin{center}

{\it J. Liouville} % replace this with your name

\end{center}

\subsection{Introduction}

\subsection{Glossary of Terms}

\subsection{Theorems}

\subsection{Examples}

%~~~~~~~~~~~~~~~~~~~~~~~~~~~~~~~~~~~~~~~~~~~~~~~~
%~~~~~~~~~~~~~~~~ Chapter 4 ~~~~~~~~~~~~~~~~~~~~~~~
%~~~~~~~~~~~~~~~~~~~~~~~~~~~~~~~~~~~~~~~~~~~~~~~~

\section{Fermat's Little Theorem} % Replace this with your chapter number and title. 

\setcounter{equation}{0}
\setcounter{theorem}{0}

\addcontentsline{toc}{subsection}{{\it P. Fermat}} % Replace this with your name and chapter title. 
\begin{center}
{\it P. Fermat} % replace this with your name
\end{center}

\subsection{Introduction}

\subsection{Glossary of Terms}

\subsection{Theorems}

\subsection{Examples}

%~~~~~~~~~~~~~~~~~~~~~~~~~~~~~~~~~~~~~~~~~~~~~~~~
%~~~~~~~~~~~~~~~~ Chapter 5 ~~~~~~~~~~~~~~~~~~~~~~~
%~~~~~~~~~~~~~~~~~~~~~~~~~~~~~~~~~~~~~~~~~~~~~~~~

\section{Polynomial Congruences} % Replace this with your chapter number and title. 

\setcounter{equation}{0}
\setcounter{theorem}{0}

\addcontentsline{toc}{subsection}{{\it J. Lagrange}} % Replace this with your name and chapter title. 
\begin{center}
{\it J. Lagrange} % replace this with your name
\end{center}

\subsection{Introduction}

\subsection{Glossary of Terms}

\subsection{Theorems}

\subsection{Examples}

%~~~~~~~~~~~~~~~~~~~~~~~~~~~~~~~~~~~~~~~~~~~~~~~~
%~~~~~~~~~~~~~~~~ Chapter 6 ~~~~~~~~~~~~~~~~~~~~~~~
%~~~~~~~~~~~~~~~~~~~~~~~~~~~~~~~~~~~~~~~~~~~~~~~~

\section{Quadratic Reciprocity} % Replace this with your chapter number and title. 

\setcounter{equation}{0}
\setcounter{theorem}{0}

\addcontentsline{toc}{subsection}{{\it D. Hilbert}} % Replace this with your name and chapter title. 
\begin{center}
{\it D. Hilbert} % replace this with your name
\end{center}

\subsection{Introduction}

\subsection{Glossary of Terms}

\subsection{Theorems}

\subsection{Examples}




\end{document}















